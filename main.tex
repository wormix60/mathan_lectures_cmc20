\documentclass[a4paper,12pt]{article}  

\usepackage[T2A]{fontenc}
\usepackage[utf8]{inputenc}
\usepackage[russian]{babel}
\usepackage{amsmath}
\usepackage{amsthm}
\usepackage{amssymb}
\usepackage{ulem}
\usepackage{indentfirst}
\usepackage{pgfplots}
\usepackage{tikz}
\usepackage{graphicx}
\usepackage[hidelinks]{hyperref}

\hypersetup{
    linktoc=all
}

\pgfplotsset{compat=1.15}

\newtheorem{definition}{Определение}[subsection]
\newtheorem{theorem}{Теорема}[subsection]
\newtheorem{statement}{Утверждение}[subsection]
\newtheorem{lemma}[theorem]{Лемма}
\newtheorem{conseq}{Следствие}[theorem]
\newtheorem{remark}{Замечание}[theorem]
\newtheorem{corollary}{Пример}[subsection]
\newtheorem{method}{Метод}[subsection]

\numberwithin{equation}{subsection}

\begin{document}  
\pagenumbering{Alph}
\begin{titlepage}
\newpage
\begin{center}
\Large{\textsc{Лекции по матанализу.}}

\LaTeX
\end{center}
\vspace{2cm}
\begin{center}
Работяга из 206.
\end{center}

\vspace{\fill}
\begin{center}
2020.
\end{center}
\end{titlepage}
\pagenumbering{arabic}
\includegraphics[width=\textwidth]{me}
\newpage

\tableofcontents
\newpage

\input{splits1}
\input{splits2}

\end{document}